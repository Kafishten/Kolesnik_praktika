%% -*- coding: utf-8 -*-
\documentclass[12pt,a4paper]{scrartcl} 
\usepackage[utf8]{inputenc}
\usepackage[english,russian]{babel}
\usepackage{indentfirst}
\usepackage{misccorr}
\usepackage{graphicx}
\usepackage{amsmath}
\begin{document}
\begin{titlepage}
		\begin{center}
			\large
			МИНИСТЕРСТВО НАУКИ И ВЫСШЕГО ОБРАЗОВАНИЯ РОССИЙСКОЙ ФЕДЕРАЦИИ
			
			Федеральное государственное бюджетное образовательное учреждение высшего образования
			
			\textbf{АДЫГЕЙСКИЙ ГОСУДАРСТВЕННЫЙ УНИВЕРСИТЕТ}
			\vspace{0.25cm}
			
			Инженерно-физический факультет
			
			Кафедра автоматизированных систем обработки информации и управления
			\vfill

			\vfill
			
			\textsc{Отчет по практике}\\[5mm]
			
			{\LARGE Программаная реализация численного метода \textit{Найти определитель матрицы.}}
			\bigskip
			
			1 курс, группа 1ИВТ
		\end{center}
		\vfill
		
		\newlength{\ML}
		\settowidth{\ML}{«\underline{\hspace{0.7cm}}» \underline{\hspace{2cm}}}
		\hfill\begin{minipage}{0.5\textwidth}
			Выполнил:\\
			\underline{\hspace{\ML}} А.\,Е.~Колесник\\
			«\underline{\hspace{0.7cm}}» \underline{\hspace{2cm}} 2024 г.
		\end{minipage}%
		\bigskip
		
		\hfill\begin{minipage}{0.5\textwidth}
			Руководитель:\\
			\underline{\hspace{\ML}} С.\,В.~Теплоухов\\
			«\underline{\hspace{0.7cm}}» \underline{\hspace{2cm}} 2024 г.
		\end{minipage}%
		\vfill
		
		\begin{center}
			Майкоп, 2024 г.
		\end{center}
	\end{titlepage}
 
\section{Введение.}
\label{sec:intro}
% Что должно быть во введении
Задание:
\begin{description}
    Найти определитель квадратной матрицы порядка n (м. Гаусса).
\end{description}
\section{Ход работы}
\label{sec:exp}
\subsection{Код приложения}
\label{sec:exp:code}
\begin{verbatim}
import random
m=[]
det=1
print("Ввод данных\n")
choose=int(input("Введите 1, чтобы матрица \nзаполнилась случайными числами: "))
n=int(input("\nЗадание порядка матрицы: "))

if choose==1: 
    for i in range(n):
        m.append(random.sample(range(1,10),n))
else:
    for i in range(n):
        m.append([])
        print("Ввод значений в ",i+1,"строку матрицы: ")
        for i2 in range(n):
            m[i].append(0)
            m[i][i2]=int(input(": "))

print('\nИсходная матрица:') 
for i in range(n):
    print(m[i])

for i in range(n-1): 
    for i1 in range(n-1-i):
        k=m[i1+1+i][i]/m[i][i]
        for i2 in range(n-i):
            m[i1+1+i][i2+i]-=m[i][i2+i]*k

for i in range(n):
    det*=m[i][i]

print('\n\nРезультат:') 
print("det A = ",round(det))
\end{verbatim}
\section{Блок-схема и примеры работы программы:}
\label{sec:pictures}
\begin{figure}[h]
	\centering
	\includegraphics[width=0.47\textwidth]{2.jpg}
	\caption{Пример работы программы со случайной генерацией матрицы.}\label{fig:par}
\end{figure}
\begin{figure}[h]
	\centering
	\includegraphics[width=0.47\textwidth]{3.jpg}
	\caption{Пример работы программы с ручным вводом значений.}\label{fig:par}
\end{figure}





\begin{thebibliography}{9}
\bibitem{Lvovsky-2003}Львовский С.М. Набор и верстка в системе \LaTeX{}. \newblock --- 3-е издание, исправленное и дополненное, 2003 г.
\bibitem{Voroncov-2005}Воронцов К.В. \LaTeX{} в примерах. 2005 г.
\end{thebibliography}
\end{document}
